\documentclass[12pt]{ctexrep}
\usepackage{graphicx} %如需在文档中插入图片
\usepackage{geometry} %调整页面设置,如页边距
\usepackage{float} %强制修改图片插入位置
\usepackage{enumerate} %有序列表
\usepackage{booktabs} %插入表格
\usepackage{amsmath} %长公式手动换行
\usepackage{lmodern}%解决字形问题
\usepackage{fancyhdr}%页眉页脚设置
\usepackage{multirow}%合并单元格
\usepackage{caption}%修改图表caption的格式
\usepackage{pdfpages}%插入pdf
\usepackage{rotating}%使表格旋转90度纵向放置
\usepackage{afterpage}%使用\clearpage指令需要的宏包
\usepackage{titlesec}%设置标题样式
\usepackage{fontspec}
\usepackage{ctex}
\usepackage{CJKutf8}
\usepackage{setspace}
\usepackage{tocloft}%设置目录行间距


\captionsetup[figure]{labelsep=space}{}%使用空格作为图表编号和caption之间的分隔符
\captionsetup[table]{labelsep=space}%设置表格标题格式
%-----------------设置页眉页脚----------------
\pagestyle{fancy}
\fancyhf{}
\fancyhead[L]{}
\fancyhead[R]{}
\chead{青岛理工大学毕业设计(论文)}
\cfoot{第 \thepage 页}
%----------------标题日期作者-----------------
\title{A \LaTeX{} sample}
\author{JerryLF7}
\date{\today}
%-----------------页边距-----------------
\geometry{a4paper, top=2.54cm, bottom=2.54cm, left=3.18cm, right=3.18cm}

%--------------一级级标题格式-------------
\titleformat{\section}
    {\normalfont\heiti\zihao{3}\bfseries}{\thechapter}{1em}{}
%----------------二级标题格式---------------
\titleformat{\section}
    {\normalfont\heiti\zihao{4}\bfseries}{\thesection}{0.5em}{}
%----------------三级标题格式---------------
\titleformat{\subsection}
    {\normalfont\bfseries}{\hspace{2em}\thesubsection}{0.5em}{}
%----------------四级标题格式---------------
\setcounter{secnumdepth}{4}%添加四级标题的计数器
\titleformat{\subsubsection}
    {\normalfont}{\hspace{2em}\thesubsubsection}{0.5em}{}%设置四级标题的序号和样式

%-----------------设置标题字体------------------
\ctexset{
  chapter = {
    format = \centering\heiti\zihao{3},
  },
  section = {
    format = \heiti\zihao{4},
  },
  subsection = {
    format = \heiti\zihao{-4},
  },
}

%--------------------设置正文字体---------------
\ctexset{
  paragraph = {
    format = \songti\zihao{-4},
  },
}

\begin{document}\sloppy %over full错误时,自动处理行宽度超过页面宽度的情况

\maketitle
%--------------中文摘要和关键词------------
\begin{center}
    \zihao{3}\heiti 摘\hspace{1em}要
  \end{center}
  
  % 摘要内容
  \vspace{1em}
  \begin{spacing}{1.5}
    \zihao{-4}\songti 
    在这里填写中文摘要
    
  \end{spacing}
  
  % 关键词
  \vspace{1em}
  \begin{spacing}{1}
    \zihao{-4}\heiti 关键词:\zihao{-4}\songti 模板,摘要
  \end{spacing}
  

%----------------英文摘要-----------------
\newpage
\newfontfamily\englishtitlefont{Times New Roman}

\begin{center}
  \fontsize{16}{24}\selectfont\englishtitlefont Abstract
\end{center}

\vspace{1em}
\fontsize{12}{18}\selectfont
English abstract should be filled here.\\

\textbf{KEY WORDS:} Template, abstract



%-------------------目录-----------------
\newpage
\renewcommand{\contentsname}{\centering 目录} % 将目录名称居中对齐
{\centering \tableofcontents}

\setcounter{page}{0}
\thispagestyle{empty}

%------------------正文--------------------
\chapter{一级标题}
    \section{二级标题}
      \subsection{三级标题}
            在这里填写正文。

            \begin{figure}[h]
                \centering
                \includegraphics[width=0.7\textwidth]{images/sample.jpg}
                \caption{在这里填写图片描述}
            \end{figure}

\chapter{常用样式示例}
    单行公式:
    $$M=\frac{1}{8}ql^2=0.125\times 1.5\times 1.95^2=0.71kN\cdot $$

    多行公式:
    $$I_{eq}^s=\frac{I_s^u+I_s^c}{2}=\frac{2.27\times10^8+1.37\times10^8}{2}=1.82\times10^8mm^4$$
    $$B^s=E_c I_{eq}^s=3.00\times 10^4 \times 1.82 \times 10^8=5.46\times10^{12}N\cdot mm^2$$

    多行公式(等号对齐):
    \begin{eqnarray}
      M_u&=&f_c bx(h_0-\frac{x}{2})  \nonumber    \\
      ~&=&14.3\times 1000\times 16.67\times (113-\frac{16.67}{2})\times 10^{-6} \nonumber    \\
      ~&=&24.95kN\cdot m >M^+=11.5kN\cdot m \nonumber
    \end{eqnarray}

    数字序号:
    \begin{enumerate}
      \item 第一项
      \item 第二项
    \end{enumerate}

    多级数字序号:
    \begin{enumerate}
      \item 一级序号
      \item 一级序号
      \begin{enumerate}[(1)]
          \item 二级序号
          \item 二级序号
      \end{enumerate}
    \end{enumerate}

    简单表格(共2列):

    \begin{table}[H]
      \centering
      \caption{压型钢板相关参数}
      \begin{tabular}{lr}
          \toprule
          列1表头&列2表头\\
          \midrule
          列1内容&列2内容\\
          列1内容&列2内容\\
          列1内容&列2内容\\
          列1内容&列2内容\\
          列1内容&列2内容\\
          \bottomrule
      \end{tabular}
    \end{table}

    罗马数字:
    
    \uppercase\expandafter{\romannumeral1},\uppercase\expandafter{\romannumeral2},
    \uppercase\expandafter{\romannumeral3}

\newpage % 可选,另起一页
\chapter*{致谢} % 使用 * 使此部分不计入章节编号
%\addcontentsline{toc}{chapter}{致谢} 
%如果需要将致谢添加到目录中,则取消注释
    在这里填写致谢内容。

\newpage

\chapter*{参考文献}
%\addcontentsline{toc}{chapter}{参考文献}
%如果需要将参考文献添加到目录中,则取消注释
    \begin{enumerate}[{[}1{]}]
      \item 参考文献1;
      \item 参考文献2;
    \end{enumerate}

\end{document}